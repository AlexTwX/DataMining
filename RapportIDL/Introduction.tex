\section{Introduction}
Lors du développement de logiciel, il est fréquent de voir apparaitre des bugs, due à de nombreux facteurs tel que l'expérience du développeur ou la maturité d'un framework. C'est dans ce but que de nombreux sites proposent leurs services afin de permettrent aux développeurs de pouvoir partager leurs connaissances. Il existe également des sites spécialement conçus afin de pouvoir exposer les problèmes rencontrés lors d'un développement comme par exemple stackoverflow.com. 

Une stack trace (ou trace d'appels en français) est un affichage de la pile d'éxécution à un instant précis de l'éxecution d'un programme. Bien que pouvant être affichés à n'importe quel moment par le dévelopeur, cette dernière est le plus souvent utilisé lors de la phase de débuggage. C'est en général ce qui est affiché dans le terminal par l'interpréteur lorsqu'un logiciel rencontre une erreur grave. De nombreux développeurs utilisent celles-ci pour découvrir l'origine de l'erreur. 

C'est sur ce constat que nous allons réalisés des statistiques concernant les pratiques des utilisateurs du site stackoverflow.com. Nous allons nous concentrer sur la présence des stack traces dans leurs questions et leur influence sur les réponses apportées (quantité, pertinence, résolution..).

