\documentclass[11pt,a4paper]{article}
\usepackage[utf8]{inputenc}
\usepackage[T1]{fontenc}

\usepackage{listings}
\usepackage{color}

\definecolor{mygreen}{rgb}{0,0.6,0}
\definecolor{mygray}{rgb}{0.5,0.5,0.5}
\definecolor{mymauve}{rgb}{0.58,0,0.82}

\usepackage[francais]{babel}
\usepackage{caption}
\usepackage[left=2cm,right=2cm,top=2cm,bottom=2cm]{geometry}
\usepackage{graphicx}
\usepackage{helvet} % Pour le texe
\usepackage[hidelinks]{hyperref}
\usepackage{listings}
\usepackage{url}
\usepackage{algpseudocode}

\renewcommand{\familydefault}{\sfdefault}

\usepackage{inconsolata} % Pour les listings

\author{Grégory LEFER \and Alexandre FRANCOIS}
\title{Analyse de la présence des stacktrace python sur Stackoverflow.com}
\date{02/12/2014}

\lstset{ % Adapted from http://en.wikibooks.org/wiki/LaTeX/Source_Code_Listings
  backgroundcolor=\color{white},   % choose the background color; you must add \usepackage{color} or \usepackage{xcolor}
  basicstyle=\footnotesize\ttfamily,        % the size of the fonts that are used for the code
  breakatwhitespace=false,         % sets if automatic breaks should only happen at whitespace
  breaklines=true,                 % sets automatic line breaking
  captionpos=b,                    % sets the caption-position to bottom
  commentstyle=\color{mygreen},    % comment style
  extendedchars=true,              % lets you use non-ASCII characters; for 8-bits encodings only, does not work with UTF-8
  frame=single,                    % adds a frame around the code
  keepspaces=true,                 % keeps spaces in text, useful for keeping indentation of code (possibly needs columns=flexible)
  keywordstyle=\color{blue},       % keyword style
  morekeywords={func, package, import},            % if you want to add more keywords to the set
  numbers=left,                    % where to put the line-numbers; possible values are (none, left, right)
  numbersep=5pt,                   % how far the line-numbers are from the code
  numberstyle=\tiny\color{mygray}, % the style that is used for the line-numbers
  rulecolor=\color{black},         % if not set, the frame-color may be changed on line-breaks within not-black text (e.g. comments (green here))
  showspaces=false,                % show spaces everywhere adding particular underscores; it overrides 'showstringspaces'
  showstringspaces=false,          % underline spaces within strings only
  showtabs=false,                  % show tabs within strings adding particular underscores
  stepnumber=1,                    % the step between two line-numbers. If it's 1, each line will be numbered
  stringstyle=\color{mymauve},     % string literal style
  tabsize=2,                       % sets default tabsize to 2 spaces
  title=\lstname                   % show the filename of files included with \lstinputlisting; also try caption instead of title
}

\begin{document}
\maketitle
\tableofcontents

\section{Introduction}
Lors du développement de logiciel, il est fréquent de voir apparaitre des bugs, due à de nombreux facteurs tel que l'expérience du développeur ou la maturité d'un framework. C'est dans ce but que de nombreux sites proposent leurs services afin de permettre aux développeurs de pouvoir partager leurs connaissances. Il existe également des sites spécialement conçus afin de pouvoir exposer les problèmes rencontrés lors d'un développement comme par exemple stackoverflow.com. 

Une stack trace (ou trace d'appels en français) est un affichage de la pile d'éxécution à un instant précis de l'éxecution d'un programme. Bien que pouvant être affichée à n'importe quel moment par le dévelopeur, cette dernière est le plus souvent utilisée lors de la phase de débuggage. C'est en général ce qui est affiché dans le terminal par l'interpréteur lorsqu'un logiciel rencontre une erreur grave. De nombreux développeurs utilisent celle-ci pour découvrir l'origine de l'erreur. 

C'est sur ce constat que nous allons réaliser des statistiques concernant les pratiques des utilisateurs du site stackoverflow.com. Nous allons nous concentrer sur la présence des stack traces dans leurs questions et leurs influences sur les réponses apportées (quantité, pertinence, résolution..).

\section{Développement}
\subsection{Reconnaissance de stacktrace python}
\subsection{Stackoverflow datadump}
\subsection{Parser crash dataset}
\subsection{Parser question avec crash}
\subsection{Parser question sans crash}

\section{Exploitation des datasets}
\subsection{Bucket, Xbase et Xpath}
\subsection{CrashDataset}
\subsection{Question avec crash}
\subsection{Question sans crash}

\section{Interprétation des données}
\subsection{Stat CrashDataset}
\subsection{Stat Question avec crash}
\subsection{Stat Question sans crash}
\section{Conclusion}

% \begin{figure}
% \begin{lstlisting}
% package main

% import "fmt"

% func main() {
% 	fmt.Println("Hello, World!")
% }
% \end{lstlisting}
% \caption{Votre premier programme en go}
% \label{code:sample1}
% \end{figure}

%La figure \ref{code:sample1} à la page \pageref{code:sample1} est un exemple de programme go.

%Le programme de la formation n'est pas à jour\cite{fil}.

\bibliography{references}
\bibliographystyle{plain}
\end{document}
