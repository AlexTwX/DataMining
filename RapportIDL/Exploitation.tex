\section{Exploitation des datasets}
Cette section présente les differentes manières utilisées pour l'exploitation de ces datasets
\subsection{Buckets, baseX, XPath}
Le dataset contenant l'ensemble des stacktraces à été diviser en bucket en fonction du type d'erreur ou exception survenue.
Les datasets contenant les questions et leurs réponses associées ont été exploité grâce au logiciel baseX qui  est un système de gestion de base de données XML native et légère et est spécialisé dans le stockage, le requêtage et la visualisation de larges documents et collections de documents XML.
Pour effectuer nos differentes requêtes le langage XPath à été utilisé. c'est un langage pour localiser une portion d'un document XML simple d'emploi.

\subsection{Stacktrace dataset}
\subsection{Question avec stacktrace}
\begin{figure}
\begin{lstlisting}
#Nombre de questions
    //Question
    Resultat: 12492
#Nombre de questions contenant une reponse acceptee
    //Question[AcceptedAnswer/Score]
    Resultat: 7167
#Nombre de questions ne contenant pas de reponse  acceptee mais ayant des reponse
    //Question[not(AcceptedAnswer/Score) and Reponses//Score]
    Resultat: 3796
#Nombre de questions ne contenant pas de reponse acceptee mais ayant des reponse avec un score positif
    //Question[not(AcceptedAnswer/Score) and Reponses//Score > 0] 
    Resultat: 2283
#Nombre de questions contenant une reponse acceptee mais ayant des reponse avec un score plus eleve
    //Question[Reponses//Score  > AcceptedAnswer/Score]
    Resultat: 713
#Nombre de reponse contenant une stacktrace
    //Question[Reponses//Stack]
    Resultat: 99
\end{lstlisting}
\caption{Requêtes effectué sur le dataset contenant les questions avec les stacktraces}
\label{code:resultatAvecStack}
\end{figure}

\subsection{Question sans stacktrace}
\begin{figure}
\begin{lstlisting}
#Nombre de questions
    //Question
    Resultat: 345217
#Nombre de questions contenant une reponse acceptee
    //Question[AcceptedAnswer/Score]
    Resultat: 215288
#Nombre de questions ne contenant pas de reponse acceptee mais ayant des reponse
    //Question[not(AcceptedAnswer/Score) and Reponses//Score]
    Resultat: 98154
#Nombre de questions ne contenant pas de reponse acceptee mais ayant des reponse avec un score positif
    //Question[not(AcceptedAnswer/Score) and Reponses//Score > 0] 
    Resultat: 62907
#Nombre de questions contenant une reponse acceptee mais ayant des reponse avec un score plus eleve
    //Question[Reponses//Score  > AcceptedAnswer/Score]
    Resultat: 29485
#Nombre de reponse contenant une stacktrace
    //Question[Reponses//Stack]
    Resultat: 626
#Nombre de question ne contenant pas de reponse
    //Question[not(Reponses/Reponse)]
    Resultat: 131175
\end{lstlisting}
\caption{Requêtes effectué sur le dataset contenant les questions sans stacktraces}
\label{code:resultatSansStack}
\end{figure}